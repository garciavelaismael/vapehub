\documentclass[a4paper]{article} %Formato de plantilla que vamos a utilizar
\usepackage[utf8]{inputenc}
\usepackage[spanish]{babel}
\usepackage[a4paper, total={6.5in, 9.5in}]{geometry}
\usepackage{graphicx} % Insertar imágenes
\usepackage{hyperref} % Insertar url
\usepackage{listings} % Insertar código
\usepackage[table,xcdraw]{xcolor} % Detección de colores
\usepackage{fancyhdr} % Crear cabeceros

% Declaración de colores
\definecolor{blue}{HTML}{03aaf9}

% Declaración de variables
\newcommand{\logoPunta}{img/LogoPuntaDelVerde.png} % Logo del punta
\newcommand{\logoPractica}{img/vape.png} % Logo de la práctica
\newcommand{\nombrePractica}{VapeHub} % Nombre de la práctica
\newcommand{\numPractica}{Angular} % Número de la práctica
\newcommand{\fecha}{14/02/2022} % Fecha de entrega

\lstdefinelanguage{Typescript}{
keywords={typeof, new, true, false, catch, function, return, null, catch, get, witch, var, if, else},
keywordstyle=\color{blue}\bfseries,
ndkeywords={class, export, boolean, throw, implements, import, this},
ndkeywordstyle=\color{darkgray}\bfseries,
identifierstyle=\color{black},
sensitive=false,
comment=[l]{//},
morecomment=[s]{/*}{*/},
commentstyle=\color{purple}\ttfamily,
stringstyle=\color{red}\ttfamily,
morestring=[b]',
morestring=[b]"
}

\lstset{
language=Typescript,
extendedchars=true,
basicstyle=\footnotesize\ttfamily,
showstringspaces=false,
showspaces=false,
numbers=left,
numberstyle=\footnotesize,
numbersep=9pt,
tabsize=2,
breaklines=true,
showtabs=false,
captionpos=b
}

% Header y footer
\pagestyle{fancy}
\fancyhf{}
\rhead{\includegraphics[width=15cm]{\logoPunta}}
\rfoot{\thepage}
\lfoot{Ismael García Vela\par\fecha}
\renewcommand{\headrulewidth}{0pt}
\renewcommand{\footrulewidth}{1pt}
\setlength\headheight{40pt} % quitar aviso cabecero

% Comienzo del documento
\begin{document}
% Portada
\begin{titlepage}
	\centering
	\vspace*{7cm}
	\includegraphics[width=0.3\textwidth]
	{\logoPractica}\par\vspace{1cm}
	{\LARGE\textbf{Proyecto \numPractica}\par\vspace{0.2cm}}
	{\Huge\bfseries\textcolor{gray}{\nombrePractica}}\par\vspace{1cm}
\end{titlepage}

% Índice
\tableofcontents

% Documentación
\newpage
\section{Introducción}
Proyecto de una aplicación que trata datos mostrados de forma visual gracias a Angular
usando la librería de componentes de Angular Material. Es una aplicación destinada
a un empleado el cuál es administrador o es encargado de un punto de venta.
\par\vspace{0.5cm}
Esta alojada en Heroku desplegada por el repositorio de github. Usa una API la cuál
conecta con una base de datos alojada en Mongo Atlas, desplegada en Heroku al igual que la APP.
\par\vspace{0.5cm}
Primero veremos las clases y modelos usados, y ya luego pasaremos a la aplicación.
\clearpage
\section{Clases}
\subsection{Persona}
\subsubsection{Persona}
\begin{lstlisting}[language=Typescript]
	export class Persona {
  protected _id: string;
  protected _nombre: string;
  protected _direccion: { calle: string; numero: number };
  protected _telefono: number;
  protected _email: string;
  constructor(
    id: string,
    nombre: string,
    direccion: { calle: string; numero: number },
    telefono: number,
    email: string,
  ) {
    this._id = id;
    this._nombre = nombre;
    this._direccion = direccion;
    this._telefono = telefono;
    this._email = email;
  }
  get id() {
    return this._id;
  }
  get nombre() {
    return this._nombre;
  }
  get direccion() {
    return this._direccion;
  }
  get telefono() {
    return this._telefono;
  }
  get email() {
    return this._email;
  }

  todo() {
    return `ID: ${this._id}, 
    Nombre: ${this._nombre}, 
    Direccion: ${this._direccion}, 
    Telefono: ${this._telefono}, 
    Email : ${this._email}`;
  }
}
\end{lstlisting}\clearpage
\subsubsection{Cliente}
\begin{lstlisting}[language=Typescript]
	import { Persona } from "./persona";
export class Cliente extends Persona {
  protected _socio: boolean;
  constructor(
    id: string,
    nombre: string,
    direccion: { calle: string; numero: number },
    telefono: number,
    email: string,
    socio: boolean,
  ) {
    super(id, nombre, direccion, telefono, email);
    this._socio = socio;
  }
  get socio() {
    return this._socio;
  }
}
\end{lstlisting}
\subsubsection{Empleado}
\begin{lstlisting}[language=Typescript]
	import { Persona } from "./persona";
export class Empleado extends Persona {
  protected _ventas: number;
  protected _horas: number;
  constructor(
    id: string,
    nombre: string,
    direccion: { calle: string; numero: number },
    telefono: number,
    email: string,
    ventas: number,
    horas: number,
  ) {
    super(id, nombre, direccion, telefono, email);
    this._ventas = ventas;
    this._horas = horas;
  }
  get ventas() {
    return this._ventas;
  }
  salario(): number {
    let salario: number;
    let base: number = this._horas * 8;
    if (this._ventas > 5) { salario = base * 1.02 } 
    else if (this._ventas > 10) {
      salario = base * 1.03
    } else if (this._ventas > 20) {
      salario = base * 1.05
    } else {
      salario = base
    }
    return Math.round(salario)
  }
}
\end{lstlisting}\clearpage
\subsection{Producto}
\subsubsection{Producto}
\begin{lstlisting}[language=Typescript]
	export class Producto {
		protected _id: string;
		protected _nombreProd: string;
		protected _marca: string;
		protected _coste: number;
	  
		constructor(id: string, nombreProd: string, marca: string, coste: number) {
		  this._id = id;
		  this._nombreProd = nombreProd;
		  this._marca = marca;
		  this._coste = coste;
		}
	  
		get id() {
		  return this._id;
		}
		get nombreProd() {
		  return this._nombreProd;
		}
		get marca() {
		  return this._marca;
		}
		get coste() {
		  return this._coste;
		}
	  
		todoProd() {
		  return `ID: ${this._id}, 
			Nombre de producto: ${this._nombreProd}, 
			Marca: ${this._marca}
			coste ${this._coste}`;
		}
	  }	  
\end{lstlisting}
\subsubsection{Líquido}
\begin{lstlisting}
	export class Liquido extends Producto {

  protected _sabor: string;
  protected _nicotina: number;

  constructor(
    id: string,
    nombreProd: string,
    marca: string,
    sabor: string,
    nicotina: number,
    coste: number
  ) {
    super(id, nombreProd, marca, coste);

    this._sabor = sabor;
    this._nicotina = nicotina;
  }
  get sabor() {
    return this._sabor;
  }
  get nicotina() {
    return this._nicotina;
  }
}
\end{lstlisting}\clearpage
\subsubsection{Dispositivo}
\begin{lstlisting}
	export class Dispositivo extends Producto {
  protected _potencia: number;
  protected _bateria: number;

  constructor(
    id: string,
    nombreProd: string,
    marca: string,
    potencia: number,
    bateria: number,
    coste: number
  ) {
    super(id, nombreProd, marca, coste);

    this._potencia = potencia;
    this._bateria = bateria;
  }
  get potencia() {
    return this._potencia;
  }
  get bateria() {
    return this._bateria;
  }
}
\end{lstlisting}\clearpage
\section{Compra}
\begin{lstlisting}[language=Typescript]
	export class Compra {
    protected _id: string;
    protected _idCliente: string;
    protected _coste: number;
    protected _idProducto: string;

    constructor(id: string, idCliente: string, coste: number, idProducto: string) {
      this._id = id;
      this._idCliente = idCliente;
      this._coste = coste;
      this._idProducto = idProducto;
    }
    get getId() {
      return this._id;
    }
    get getCliente() {
      return this._idCliente;
    }
    get getProductos() {
      return this._idProducto;
    }
  
    todoCompra() {
      return `ID: ${this._id}, 
        : nombreCliente: ${this._idCliente}, 
        coste: ${this._coste},
        productos:  ${this._idProducto}`;
    }
  }
\end{lstlisting}\clearpage
\section{Tipos}
\subsection{Persona}
\begin{lstlisting}[language=Typescript]
export type tCliente = {
  id: string | null;
  nombre: string | null;
  calle: string;
  numero: number;
  telefono: number | null;
  email: string | null;
  socio: Boolean | null;
};

export type tEmpleado = {
  id: string | null;
  nombre: string | null;
  calle: string;
  numero: number;
  telefono: number | null;
  email: string | null;
  ventas: number | null;
  horas: number | null;
};

export type tEmpleado2 = {
  _id: string;
  _nombre: string;
  _direccion: {
    calle: string;
    numero: number;
  };
  _telefono: number;
  _email: string;
  _ventas: number;
  _horas: number;
};

export type tSalario = {
  _id: string | null;
  _nombre: string | null;
  _salario: number | null;
};
\end{lstlisting}
\subsection{Producto}
\begin{lstlisting}[language=Typescript]
export type tDispositivo = {
  id: string | null;
  nombreProd: string | null;
  marca: string | null;
  potencia: number | null;
  bateria: number | null;
  coste: number | null;
};
export type tLiquido = {
  id: string | null;
  nombreProd: string | null;
  marca: string | null;
  sabor: string | null;
  nicotina: number | null;
  coste: number | null;
};
\end{lstlisting}\clearpage
\subsection{Compra}
\begin{lstlisting}[language=Typescript]
export type tCompra = {
	id: string | null;
	idCliente: string | null;
	idProducto: string | null;
	coste: number | null;
	};
\end{lstlisting}\clearpage
\section{Rutas}
Funciones usadas en el routing:
\begin{itemize}
	\item Listar:
\begin{lstlisting}[language=Typescript]
	// Listar empleados
  private getEmpleados = async (req: Request, res: Response) => {
    await db.conectarBD()
      .then(async (mensaje) => {
        const valor = req.params.id
        console.log(mensaje)
        const query = await EmpleadoDB.find();
        res.json(query)
      })
      .catch((mensaje) => {
        res.send(mensaje)
      })
    db.desconectarBD()
  }

  // Listar clientes
  private getClientes = async (req: Request, res: Response) => {
    await db.conectarBD()
      .then(async (mensaje) => {
        const valor = req.params.id
        console.log(mensaje)
        const query = await ClienteDB.find();
        res.json(query)
      })
      .catch((mensaje) => {
        res.send(mensaje)
      })
    db.desconectarBD()
  }

  // Listar dispositivos
  private getDispositivos = async (req: Request, res: Response) => {
    await db.conectarBD()
      .then(async (mensaje) => {
        const valor = req.params.id
        console.log(mensaje)
        const query = await DispositivoDB.find();
        res.json(query)
      })
      .catch((mensaje) => {
        res.send(mensaje)
      })
    db.desconectarBD()
  }

  // Listar liquidos
  private getLiquidos = async (req: Request, res: Response) => {
    await db.conectarBD()
      .then(async (mensaje) => {
        const valor = req.params.id
        console.log(mensaje)
        const query = await LiquidoDB.find();
        res.json(query)
      })
      .catch((mensaje) => {
        res.send(mensaje)
      })
    db.desconectarBD()
  }
\end{lstlisting}\clearpage
	\item Añadir:
\begin{lstlisting}[language=Typescript]
	// Add empleado
  private addEmpleado = async (req: Request, res: Response) => {
    const { id, nombre, calle, numero, telefono, email, ventas, horas } = req.body
    await db.conectarBD()
    const dSchema = {
      _id: id,
      _nombre: nombre,
      _direccion: { calle: calle, numero: numero },
      _telefono: telefono,
      _email: email,
      _ventas: ventas,
      _horas: horas
    }
    const oSchema = new EmpleadoDB(dSchema)
    await oSchema.save()
      .then((doc: any) => res.send('Has guardado el archivo:\n' + doc))
      .catch((err: any) => res.send('Error: ' + err))

    db.desconectarBD()
  }

  // Add cliente:
  private addCliente = async (req: Request, res: Response) => {
    const { id, nombre, calle, numero, telefono, email, socio } = req.body
    await db.conectarBD()
    const dSchema = {
      _id: id,
      _nombre: nombre,
      _direccion: { calle: calle, numero: numero },
      _telefono: telefono,
      _email: email,
      _socio: socio
    }
    const oSchema = new ClienteDB(dSchema)
    await oSchema.save()
      .then((doc: any) => res.send('Has guardado el archivo:\n' + doc))
      .catch((err: any) => res.send('Error: ' + err))

    db.desconectarBD()
  }

  // Add compra
  private addCompra = async (req: Request, res: Response) => {
    const { id, idCliente, idProducto, coste } = req.body
    let fecha: Date = new Date()
    await db.conectarBD()
    const dSchema = {
      _id: id,
      _idCliente: idCliente,
      _idProducto: idProducto,
      _coste: coste,
      _fecha: fecha
    }
    const oSchema = new CompraDB(dSchema)
    await oSchema.save()
      .then((doc: any) => res.send('Has guardado el archivo:\n' + doc))
      .catch((err: any) => res.send('Error: ' + err))

    db.desconectarBD()
  }
\end{lstlisting}\clearpage
	\item Actualizar:
\begin{lstlisting}[language=Typescript]
	// Actualizar empleado
  private updateEmpleado = async (req: Request, res: Response) => {
    await db.conectarBD()
    const id = req.params.id
    const { nombre, calle, numero, telefono, email, ventas, horas } = req.body
    await EmpleadoDB.findOneAndUpdate(
      { _id: id },
      {
        _id: id,
        _nombre: nombre,
        _direccion: { calle: calle, numero: numero },
        _telefono: telefono,
        _email: email,
        _ventas: ventas,
        _horas: horas
      },
      {
        new: true,
        runValidators: true
      }
    )
      .then((doc: any) => res.send('Has guardado el archivo:\n' + doc))
      .catch((err: any) => res.send('Error: ' + err))

    await db.desconectarBD()
  }

  // Actualizar cliente
  private updateCliente = async (req: Request, res: Response) => {
    await db.conectarBD()
    const id = req.params.id
    const { nombre, calle, numero, telefono, email, socio } = req.body
    await ClienteDB.findOneAndUpdate(
      { _id: id },
      {
        _id: id,
        _nombre: nombre,
        _direccion: { calle: calle, numero: numero },
        _telefono: telefono,
        _email: email,
        _socio: socio,
      },
      {
        new: true,
        runValidators: true
      }
    )
      .then((doc: any) => res.send('Has guardado el archivo:\n' + doc))
      .catch((err: any) => res.send('Error: ' + err))

    await db.desconectarBD()
  }
\end{lstlisting}\clearpage
	\item Eliminar:
\begin{lstlisting}[language=Typescript]
	// Eliminar empleado
	private delEmpleado = async (req: Request, res: Response) => {
		await db.conectarBD()
		
		const id = req.params.id
    await EmpleadoDB.findOneAndDelete({ _id: id })
	.then((doc: any) => res.send('Eliminado correctamente.'))
      .catch((err: any) => res.send('Error: ' + err))
	  
	  await db.desconectarBD()
  }
  
  // Eliminar cliente
  private delCliente = async (req: Request, res: Response) => {
	  await db.conectarBD()

	  const id = req.params.id
	  await ClienteDB.findOneAndDelete({ _id: id })
      .then((doc: any) => res.send('Eliminado correctamente.'))
      .catch((err: any) => res.send('Error: ' + err))
	  
	  await db.desconectarBD()
	  }
\end{lstlisting}\clearpage
		\item Buscar ID:
\begin{lstlisting}[language=Typescript]
	// Buscar empleado
	private getEmpleadoId = async (req: Request, res: Response) => {
	await db.conectarBD()
	  .then(async (mensaje) => {
		const id = req.params.id
		console.log(mensaje)
		const query = await EmpleadoDB.findOne({ _id: id });
		res.json(query)
	  })
	  .catch((mensaje) => {
		res.send(mensaje)
	  })
	await db.desconectarBD()
	}
	
	// Buscar cliente
	private getClienteId = async (req: Request, res: Response) => {
	await db.conectarBD()
	  .then(async (mensaje) => {
		const id = req.params.id
		console.log(mensaje)
		const query = await ClienteDB.findOne({ _id: id });
		res.json(query)
	  })
	  .catch((mensaje) => {
		res.send(mensaje)
	  })
	await db.desconectarBD()
	}
	
	// Buscar dispositivo
	private getDispositivoId = async (req: Request, res: Response) => {
	await db.conectarBD()
	  .then(async (mensaje) => {
		const id = req.params.id
		console.log(mensaje)
		const query = await DispositivoDB.findOne({ _id: id });
		res.json(query)
	  })
	  .catch((mensaje) => {
		res.send(mensaje)
	  })
	await db.desconectarBD()
	}
	
	// Buscar liquido
	private getLiquidoId = async (req: Request, res: Response) => {
	await db.conectarBD()
	  .then(async (mensaje) => {
		const id = req.params.id
		console.log(mensaje)
		const query = await LiquidoDB.findOne({ _id: id });
		res.json(query)
	  })
	  .catch((mensaje) => {
		res.send(mensaje)
	  })
	await db.desconectarBD()
	}
\end{lstlisting}\clearpage
	\item Calcular salario:
\begin{lstlisting}[language=Typescript]
	// Calcular salario
  private getSalario = async (req: Request, res: Response) => {
    let tmpEmpleado: Empleado
    let dEmpleado: tEmpleado2
    let arraySalario: Array<tSalario> = []
    console.log('hola');
    
      await db.conectarBD()
        .then(async (mensaje) => {
          console.log(mensaje)
          const query = await EmpleadoDB.find({});
          for (dEmpleado of query) {
            tmpEmpleado = new Empleado(
              dEmpleado._id,
              dEmpleado._nombre,
              dEmpleado._direccion,
              dEmpleado._telefono,
              dEmpleado._email,
              dEmpleado._ventas,
              dEmpleado._horas)

            let salario = tmpEmpleado.salario()
            
            let oSalario: tSalario = {
              _id: null,
              _nombre: null,
              _salario: null
            }
            oSalario._id = tmpEmpleado.id
            oSalario._nombre = tmpEmpleado.nombre
            oSalario._salario = salario
            arraySalario.push(oSalario)
          }
          
          res.json(arraySalario)
        })
        .catch((mensaje) => {
          res.send(mensaje)
        })
    await db.desconectarBD()
  }
\end{lstlisting}\clearpage
\end{itemize}
Rutas usadas:
\begin{lstlisting}[language=Typescript]
	// RUTAS
	misRutas() {
	  // Listar
	  this._router.get('/empleados/salario', this.getSalario)
	  this._router.get('/empleados', this.getEmpleados)
	  this._router.get('/clientes', this.getClientes)
	  this._router.get('/empleados/:id', this.getEmpleadoId)
	  this._router.get('/clientes/:id', this.getClienteId)
	  this._router.get('/dispositivos', this.getDispositivos)
	  this._router.get('/liquidos', this.getLiquidos)
	  this._router.get('/dispositivos/:id', this.getDispositivoId)
	  this._router.get('/liquidos/:id', this.getLiquidoId)
	  // Crear
	  this._router.post('/empleados/addEmpleado', this.addEmpleado)
	  this._router.post('/clientes/addCliente', this.addCliente)
	  this._router.post('/addCompra', this.addCompra)
	  // Actualizar
	  this._router.put('/empleados/update/:id', this.updateEmpleado)
	  this._router.put('/clientes/update/:id', this.updateCliente)
	  // Eliminar
	  this._router.delete('/empleados/delete/:id', this.delEmpleado)
	  this._router.delete('/clientes/delete/:id', this.delCliente)
	}
  }
\end{lstlisting}\clearpage
\section{Aplicación}
\subsection{Servicios}
\subsubsection{Cliente}
\begin{lstlisting}[language=Typescript]
	export class ClienteService {
  baseUrl = 'https://api-vapehub.herokuapp.com/clientes'

  constructor(private http: HttpClient) { }

  getCliente(): Observable<any> {
    return this.http.get(this.baseUrl);
  }

  getClienteId(id: string): Observable<any> {
    return this.http.get(this.baseUrl + '/' + id);
  }

  deleteCliente(id: any): Observable<any> {
    return this.http.delete(this.baseUrl + '/delete/' + id, {responseType: 'text'})
  }
  
  addCliente(cliente: tCliente): Observable<any> {
    return this.http.post(this.baseUrl + '/addCliente', cliente)
  }

  editCliente(id: string, cliente: tCliente): Observable<any> {
    return this.http.put(this.baseUrl + '/update/' + id, cliente)
  }
}
\end{lstlisting}
\subsubsection{Empleado}
\begin{lstlisting}[language=Typescript]
	export class EmpleadoService {
  baseUrl = 'https://api-vapehub.herokuapp.com/empleados'

  constructor(private http: HttpClient) { }

  getEmpleado(): Observable<any> {
    return this.http.get(this.baseUrl);
  }

  getSalario(): Observable<any> {
    return this.http.get(this.baseUrl + '/salario');
  }

  getEmpleadoId(id: string): Observable<any> {
    return this.http.get(this.baseUrl + '/' + id);
  }

  deleteEmpleado(id: any): Observable<any> {
    return this.http.delete(this.baseUrl + '/delete/' + id, {responseType: 'text'})
  }
  
  addEmpleado(empleado: tEmpleado): Observable<any> {
    return this.http.post(this.baseUrl + '/addEmpleado', empleado)
  }

  editEmpleado(id: string, empleado: tEmpleado): Observable<any> {
    return this.http.put(this.baseUrl + '/update/' + id, empleado)
  }
}
\end{lstlisting}\clearpage
\subsubsection{Líquido}
\begin{lstlisting}[language=Typescript]
	export class LiquidoService {
		baseUrl = 'https://api-vapehub.herokuapp.com/liquidos'
	  
		constructor(private http: HttpClient) { }
	  
		getLiquido(): Observable<any> {
		  return this.http.get(this.baseUrl);
		}
	  
		getLiquidoId(id: string): Observable<any> {
		  return this.http.get(this.baseUrl + '/' + id);
		}
	}
\end{lstlisting}
\subsubsection{Dispositivo}
\begin{lstlisting}[language=Typescript]
	export class DispositivoService {
  baseUrl = 'https://api-vapehub.herokuapp.com/dispositivos'

  constructor(private http: HttpClient) { }

  getDispositivo(): Observable<any> {
    return this.http.get(this.baseUrl);
  }

  getDispositivoId(id: string): Observable<any> {
    return this.http.get(this.baseUrl + '/' + id);
  }
}
\end{lstlisting}
\subsubsection{Compra}
\begin{lstlisting}[language=Typescript]
	export class CompraService {
		baseUrl = 'https://api-vapehub.herokuapp.com'
	  
		constructor(private http: HttpClient) { }
	  
		addCompra(compra: tCompra): Observable<any> {
		  return this.http.post(this.baseUrl + '/addCompra', compra)
		}
	}
\end{lstlisting}
\subsubsection{Menú dinámico}
\begin{lstlisting}[language=Typescript]
	export class MenuService {

		constructor(private http: HttpClient) { }
	  
		getMenu(): Observable<Menu[]> {
		  return this.http.get<Menu[]>('./assets/data/menu.json');
		}
	}
\end{lstlisting}\clearpage
\subsection{Explicación menú}
Cómo se ve en el servicio llama a un documento .json el cuál hace que la barra de navegación
sea dinámica y de fácil extensión.\par\vspace{0.3cm}
Ese documento incluye lo siguiente:
\begin{lstlisting}
	[
    {
        "nombre": "Dashboard",
        "redirect": "/dashboard"
    },
    {
        "nombre": "Liquidos",
        "redirect": "/dashboard/liquidos"
    },
    {
        "nombre": "Dispositivos",
        "redirect": "/dashboard/dispositivos"
    },
    {
        "nombre": "Clientes",
        "redirect": "/dashboard/clientes"
    },
    {
        "nombre": "Empleados",
        "redirect": "/dashboard/empleados"
    },
    {
        "nombre": "Reportes",
        "redirect": "/dashboard/reportes"
    }
]
\end{lstlisting}
Y esos campos son importados de una interfaz menu:
\begin{lstlisting}[language=Typescript]
	export interface Menu {
    nombre: string,
    redirect: string
}
\end{lstlisting}
En el componente de la navbar importamos el servicio y la interfaz y cargamos el menu:
\begin{lstlisting}[language=Typescript]
	export class NavbarComponent implements OnInit {
	  menu: Menu[] = [];
	
	  constructor(private _menuService: MenuService) { }
	
	  ngOnInit(): void {
		this.cargarMenu();
	  }
	
	  cargarMenu(){
		this._menuService.getMenu().subscribe(data => {
		  this.menu = data;
		})
	  }
	}
\end{lstlisting}
Por último en el html de la navbar recorremos los datos y los insertamos en botones:
\begin{lstlisting}
	<button mat-button *ngFor="let item of menu" [routerLink]="item.redirect">
	{{ item.nombre }}
	</button>
\end{lstlisting}\clearpage
\subsection{Routing}
\subsubsection{app-routing}
Redirige al login directamente, y si pones cualquier ruta inexistente.\par\vspace{0.3cm}
Se crea la ruta padre llamada dashboard.
\begin{lstlisting}[language=Typescript]
	const routes: Routes = [
  { path: '', redirectTo: 'login', pathMatch: 'full' },
  { path: 'login', component: LoginComponent},
  { path: 'dashboard', loadChildren: () => import('./components/dashboard/dashboard.module').then(x => x.DashboardModule) },
  { path: '**', redirectTo: 'login', pathMatch: 'full' }
];
\end{lstlisting}
\subsubsection{dashboard-routing}
Rutas hijas:
\begin{lstlisting}[language=Typescript]
	const routes: Routes = [
		{
		  path: '', component: DashboardComponent, children: [
			{ path: '', component: InicioComponent },
			{ path: 'clientes', component: ClientesComponent },
			{ path: 'liquidos', component: LiquidosComponent },
			{ path: 'dispositivos', component: DispositivosComponent },
			{ path: 'empleados', component: EmpleadosComponent },
			{ path: 'reportes', component: ReportesComponent },
			{ path: 'crear-cliente', component: CrearClienteComponent },
			{ path: 'crear-cliente/:id', component: CrearClienteComponent },
			{ path: 'crear-empleado', component: CrearEmpleadoComponent },
			{ path: 'crear-empleado/:id', component: CrearEmpleadoComponent },
			{ path: 'compra', component: CompraComponent },
		  ]
		}
	  ];
\end{lstlisting}\clearpage
\subsection{Módulos}
\subsubsection{DashboardModule}
\begin{lstlisting}[language=Typescript]
	@NgModule({
  declarations: [
    DashboardComponent,
    InicioComponent,
    NavbarComponent,
    ReportesComponent,
    ClientesComponent,
    EmpleadosComponent,
    CrearClienteComponent,
    CrearEmpleadoComponent,
    LiquidosComponent,
    DispositivosComponent,
    CompraComponent,
],
  imports: [
    CommonModule,
    DashboardRoutingModule,
    SharedModule
  ]
})
\end{lstlisting}
\subsubsection{SharedModule}
Este SharedModule es para separar los modulos que son importados de librerías de componentes, como
los de Angular Material y los de HighCharts.
\begin{lstlisting}[language=Typescript]
	@NgModule({
  declarations: [],
  imports: [ CommonModule,
    ReactiveFormsModule,
    MatFormFieldModule,
    MatInputModule,
    MatButtonModule,
    MatSnackBarModule,
    MatProgressSpinnerModule,
    MatToolbarModule,
    MatIconModule,
    HttpClientModule,
    MatTableModule,
    MatTooltipModule,
    MatPaginatorModule,
    MatSortModule,
    MatCardModule,
    MatGridListModule,
    MatSelectModule,
    MatTabsModule,
    HighchartsChartModule ],
  exports : [ ReactiveFormsModule,
    MatFormFieldModule,
    MatInputModule,
    MatButtonModule,
    MatSnackBarModule,
    MatProgressSpinnerModule,
    MatToolbarModule,
    MatIconModule,
    HttpClientModule,
    MatTableModule,
    MatTooltipModule,
    MatPaginatorModule,
    MatSortModule,
    MatCardModule,
    MatGridListModule,
    MatSelectModule,
    MatTabsModule,
    HighchartsChartModule ]
})
\end{lstlisting}\clearpage
\subsection{Componentes}
\subsubsection{ClientesComponent}
\begin{lstlisting}[language=Typescript]
	export class ClientesComponent implements OnInit {

  loading = true;
  listClientes: Cliente[] = [];

  displayedColumns = ['id', 'nombre', 'calle', 'numero', 'telefono', 'email', 'socio', 'acciones'];
  dataSource!: MatTableDataSource<any>;

  constructor(private _clienteService: ClienteService,
    private _snackBar: MatSnackBar,
    private _router: Router) { }

  ngOnInit(): void {
    this._clienteService.getCliente()
      .pipe((first()))
      .subscribe(data => {
        this.listClientes = data,
          this.dataSource = new MatTableDataSource(this.listClientes);
        this.loading = false;
      })
  }

  applyFilter(event: Event) {
    const filterValue = (event.target as HTMLInputElement).value;
    this.dataSource.filter = filterValue.trim().toLowerCase();
  }

  deleteCliente(id: string) {
    this._clienteService.deleteCliente(id)
      .pipe((first()))
      .subscribe(data => {
        console.log(data);
        this.ngOnInit();
        this._snackBar.open('Cliente eliminado correctamente', '', {
          duration: 1500,
          horizontalPosition: 'center',
          verticalPosition: 'bottom'
        })
      },
        error => console.log(error));
  }

  addCliente() {
    this._router.navigate(['/dashboard/crear-cliente']);
  }

  editCliente(id: number) {
    this._router.navigate(['/dashboard/crear-cliente', id]);
    console.log(id);
  }
}
\end{lstlisting}\clearpage
\subsubsection{EmpleadosComponent}
\begin{lstlisting}[language=Typescript]
	export class EmpleadosComponent implements OnInit {

		loading = true;
		listEmpleados: Empleado[] = [];
	  
		displayedColumns = ['id', 'nombre', 'calle', 'numero', 'telefono', 'email', 'ventas', 'horas','acciones'];
		dataSource!: MatTableDataSource<any>;
	  
		constructor(private _empleadoService: EmpleadoService,
		  private _snackBar: MatSnackBar,
		  private _router: Router) { }
	  
		ngOnInit(): void {
		  this._empleadoService.getEmpleado().subscribe(data => {
			this.listEmpleados = data,
			  this.dataSource = new MatTableDataSource(this.listEmpleados);
			this.loading = false;
		  })
		}
	  
		applyFilter(event: Event) {
		  const filterValue = (event.target as HTMLInputElement).value;
		  this.dataSource.filter = filterValue.trim().toLowerCase();
		}
	  
		deleteEmpleado(id: number) {
		  this._empleadoService.deleteEmpleado(id). subscribe(data => {
			  console.log(data);
			  this.ngOnInit();
			},
			  error => console.log(error));
		}
	  
		addEmpleado() {
		  this._router.navigate(['/dashboard/crear-empleado']);
		}
	  
		editEmpleado(id: number) {
		  this._router.navigate(['/dashboard/crear-empleado', id]);
		  console.log(id);
		}
	  }
\end{lstlisting}\clearpage
\subsubsection{CrearClienteComponent}
\begin{lstlisting}[language=Typescript]
	export class CrearClienteComponent implements OnInit {
  clienteForm!: FormGroup;
  id!: string;
  isAddMode!: boolean;

  constructor(
    private _fb: FormBuilder,
    private _clienteService: ClienteService,
    private _aRouter: ActivatedRoute,
    private _snackBar: MatSnackBar
  ) { }

  ngOnInit(): void {
    this.id = this._aRouter.snapshot.params['id'];
    this.isAddMode = !this.id;

    this.clienteForm = this._fb.group({
      id: ['', Validators.required],
      nombre: ['', Validators.required],
      calle: ['', Validators.required],
      numero: ['', Validators.required],
      telefono: ['', Validators.required],
      email: ['', Validators.required],
      socio: ['', Validators.required]
    });

    if (!this.isAddMode) {
      this.inputCliente();
    }
  }

  onSubmit() {
    if (this.isAddMode) {
      this.addCliente();
    } else {
      this.editCliente();
    }
  }

  addCliente() {
    this._clienteService.addCliente(this.clienteForm.value)
      .pipe(first())
      .subscribe();
  }

  inputCliente() {
    this._clienteService.getClienteId(this.id).subscribe(data => {
      console.log(data);
      this.clienteForm.setValue({
        id: data._id,
        nombre: data._nombre,
        calle: data._direccion.calle,
        numero: data._direccion.numero,
        telefono: data._telefono,
        email: data._email,
        socio: data._socio,
      })
    })
  }

  editCliente() {
    this._clienteService.editCliente(this.id!, this.clienteForm.value)
      .pipe(first())
      .subscribe();
  }
}
\end{lstlisting}\clearpage
\subsubsection{CrearEmpleadoComponent}
\begin{lstlisting}[language=Typescript]
	export class CrearEmpleadoComponent implements OnInit {
		empleadoForm!: FormGroup;
		id!: string;
		isAddMode!: boolean;
	  
		constructor(
		  private _fb: FormBuilder,
		  private _empleadoService: EmpleadoService,
		  private _aRouter: ActivatedRoute,
		  private _snackBar: MatSnackBar
		) { }
	  
		ngOnInit(): void {
		  this.id = this._aRouter.snapshot.params['id'];
		  this.isAddMode = !this.id;
	  
		  this.empleadoForm = this._fb.group({
			id: ['', Validators.required],
			nombre: ['', Validators.required],
			calle: ['', Validators.required],
			numero: ['', Validators.required],
			telefono: ['', Validators.required],
			email: ['', Validators.required],
			ventas: ['', Validators.required],
			horas: ['', Validators.required]
		  });
	  
		  if (!this.isAddMode) {
			this.inputEmpleado();
		  }
		}
	  
		onSubmit() {
		  if (this.isAddMode) {
			this.addEmpleado();
		  } else {
			this.editEmpleado();
		  }
		}
	  
		addEmpleado() {
		  this._empleadoService.addEmpleado(this.empleadoForm.value)
			.pipe(first())
			.subscribe();
		}
	  
		inputEmpleado() {
		  this._empleadoService.getEmpleadoId(this.id).subscribe(data => {
			console.log(data);
			this.empleadoForm.setValue({
			  id: data._id,
			  nombre: data._nombre,
			  calle: data._direccion.calle,
			  numero: data._direccion.numero,
			  telefono: data._telefono,
			  email: data._email,
			  ventas: data._ventas,
			  horas: data._horas,
			})
		  })
		}
	  
		editEmpleado() {
		  this._empleadoService.editEmpleado(this.id!, this.empleadoForm.value)
			.pipe(first())
			.subscribe();
		}
	}
\end{lstlisting}\clearpage
\subsubsection{LiquidosComponent}
\begin{lstlisting}[language=Typescript]
	export class LiquidosComponent implements OnInit {
		loading = true;
		listLiquidos: Liquido[] = [];
		displayedColumns = ['id', 'nombre', 'marca', 'sabor', 'nicotina', 'coste'];
		dataSource!: MatTableDataSource<any>;
		
		@ViewChild(MatPaginator) paginator!: MatPaginator;
		@ViewChild(MatSort) sort!: MatSort;
		
		constructor(private _liquidoService: LiquidoService,
		  private _snackBar: MatSnackBar) { }
		
		ngOnInit(): void {
		  this._liquidoService.getLiquido()
			.pipe((first()))
			.subscribe(data => {
			  this.listLiquidos = data,
				this.dataSource = new MatTableDataSource(this.listLiquidos);
			  this.loading = false;
			})
		}
		
		ngAfterViewInit() {
		  //this.dataSource.paginator = this.paginator;
		  //this.dataSource.sort = this.sort;
		}
		
		applyFilter(event: Event) {
		  const filterValue = (event.target as HTMLInputElement).value;
		  this.dataSource.filter = filterValue.trim().toLowerCase();
		}
	}
\end{lstlisting}
\subsubsection{DispositivosComponent}
\begin{lstlisting}[language=Typescript]
	export class DispositivosComponent implements OnInit {
		loading = true;
		listDispositivos: Dispositivo[] = [];
		displayedColumns = ['id', 'nombre', 'marca', 'bateria', 'potencia', 'coste'];
		dataSource!: MatTableDataSource<any>;
		
		@ViewChild(MatPaginator) paginator!: MatPaginator;
		@ViewChild(MatSort) sort!: MatSort;
		
		constructor(private _dispositivoService: DispositivoService,
		  private _snackBar: MatSnackBar) { }
		
		ngOnInit(): void {
		  this._dispositivoService.getDispositivo()
			.pipe((first()))
			.subscribe(data => {
			  this.listDispositivos = data,
				this.dataSource = new MatTableDataSource(this.listDispositivos);
			  this.loading = false;
			})
		}
		
		ngAfterViewInit() {
		  //this.dataSource.paginator = this.paginator;
		  //this.dataSource.sort = this.sort;
		}
		
		applyFilter(event: Event) {
		  const filterValue = (event.target as HTMLInputElement).value;
		  this.dataSource.filter = filterValue.trim().toLowerCase();
		}
	}
\end{lstlisting}\clearpage
\subsubsection{CompraComponent}
\begin{lstlisting}[language=Typescript]
	export class CompraComponent implements OnInit {
		compraForm!: FormGroup;
		id!: string;
		idCliente: string = "";
		listCompra: Compra[] = [];
		coste!: number;
	  
		constructor(
		  private _fb: FormBuilder,
		  private _compraService: CompraService,
		  private _liquidoService: LiquidoService,
		  private _dispositivoService: DispositivoService,
		  private _snackBar: MatSnackBar
		) { }
	  
		ngOnInit(): void {
		  this.compraForm = this._fb.group({
			id: ['', Validators.required],
			dni: ['', Validators.required],
			idp: ['', Validators.required],
		  })
		}
	  
		onSubmit() {
		  this.addCompra();
		}
	  
		addCompra() {
		  this.id = this.compraForm.get('idp')?.value
	  
		  if (this.id >= '100') {
			this._liquidoService.getLiquidoId(this.id)
			  .subscribe(data => {
				console.log(data);
				this.coste = data._coste
	  
				const COMPRA: tCompra = {
				  id: this.compraForm.get('id')?.value,
				  idCliente: this.compraForm.get('dni')?.value,
				  idProducto: this.compraForm.get('idp')?.value,
				  coste: this.coste
				}
				this._compraService.addCompra(COMPRA)
				  .subscribe()
				this.compraForm.reset()
			  })
		  } if (this.id < '100') {
			this._dispositivoService.getDispositivoId(this.id)
			  .subscribe(data => {
				console.log(data);
				this.coste = data._coste
	  
				const COMPRA: tCompra = {
				  id: this.compraForm.get('id')?.value,
				  idCliente: this.compraForm.get('dni')?.value,
				  idProducto: this.compraForm.get('idp')?.value,
				  coste: this.coste
				}
				this._compraService.addCompra(COMPRA)
				  .subscribe()
				this.compraForm.reset()
			  })
		  }
		}
	}	  
\end{lstlisting}\clearpage
\subsubsection{ReportesComponent}
\begin{lstlisting}[language=Typescript]
	export class ReportesComponent implements OnInit {
		Highcharts: typeof Highcharts = Highcharts;
		listSalario: Salario[] = []
	  
		chartOptions: any = {
		  chart:
		  {
			backgroundColor: {
			  linearGradient: [500, 500, 500, 500],
			  stops: [
				[0, 'rgb(255, 255, 255)'],
			  ]
			},
			type: 'column'
		  },
		  title: {
			text: ''
		  },
		  xAxis: {
			categories: []
		  },
		  credits: {
			enabled: false
		  },
		  series: [{
			name: '',
			data: []
		  }]
		};
	  
		constructor(private _empleadoService: EmpleadoService) { }
	  
		ngOnInit(): void {
		  this.salarioEmpleado();
		}
	  
		salarioEmpleado(){
		  this._empleadoService.getSalario().subscribe((data) => {  
			this.listSalario = data
			this.listSalario.map((salario: any) => {
			  return new Salario(salario._id, salario._nombre, salario._salario)
			})
			const dataSeries = this.listSalario.map((x: Salario) => x._salario)
			const dataCategorias = this.listSalario.map((x: Salario) => x._nombre)
			
			this.chartOptions.title["text"] = "Salario de empleados"
			this.chartOptions.series[0]["data"] = dataSeries
			this.chartOptions.xAxis["categories"] = dataCategorias
			this.chartOptions.series["name"] = "Empleados"
	  
			Highcharts.chart("salario", this.chartOptions)
		  })
		}
	}
\end{lstlisting}\clearpage
\subsubsection{LoginComponent}
\begin{lstlisting}[language=Typescript]
	export class LoginComponent implements OnInit {
		form: FormGroup;
		loading = false;
	  
		constructor(private fb: FormBuilder, private _snackBar: MatSnackBar, private router: Router) {
		  this.form = this.fb.group({
			user: ['', Validators.required],
			password: ['', Validators.required]
		  })
		}
	  
		ngOnInit(): void {
		}
	  
		ingresar()  {
	  
		  const user = this.form.value.user;
		  const password = this.form.value.password;
	  
		  if(user == 'ismael' && password == '1234') {
			// Redireccion dashboard
			this.fakeLoading();
		  } else {
			// Mostrar error
			this.error();
			this.form.reset()
		  }
		}
		error() {
		  this._snackBar.open('Usuario o password incorrecta', '', {
			duration: 3000,
			horizontalPosition: 'center',
			verticalPosition: 'bottom'
		  })
		}
		fakeLoading() {
		  this.loading = true;
		  setTimeout(() => {
			// Redireccion dashboard
			this.router.navigate(['dashboard'])
		  }, 1500);
		}
	  }
\end{lstlisting}\clearpage
\end{document}
